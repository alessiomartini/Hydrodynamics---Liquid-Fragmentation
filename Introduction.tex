\chapter{Introduction}
This report is on liquid fragmentation, a phenomenon in fluid dynamics that refers to the transition from a macroscopic liquid volume to its dispersion into stable droplets. 
The cause of this fragmentation is the instabilities in the liquid. These give rise to corrugated ligaments that eventually break up, forming droplets of various sizes. 

The deformation and breakup of liquids is a common process in industrial applications and many natural phenomena. Examples include combustion engines, electro-sprayed paint, cosmetics, ink-jet printers, volcanic eruptions, rain, and tephra formation.

The first section of the report presents the characteristics of liquid fragmentation, providing a theoretical explanation and a description of the deformation and breakup of droplets falling in a quiescent medium.\cite{Mazi_article} Moreover, some models for the breakup of the liquid and the subsequent distribution of droplet sizes are discussed.\cite{Villermaux2007}

The second section...

The third section...

Conclusions